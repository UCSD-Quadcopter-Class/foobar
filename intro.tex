%auto-ignore
\section{Introduction}
\label{sec:introduction}

Persistent main memory (PM) is the first new memory technology to arrive in
the memory hierarchy since the appearance of DRAM in the early 1970's.  PM offers
numerous potential benefits including improved memory system capacity,
lower-latency and higher-bandwidth relative to disk-based storage, and a unified programming model for
persistent and volatile program state.  However, it also poses a host of novel
challenges.  For instance, it requires memory controller and ISA support, new operating
system facilities, and it places large, new burdens on programmers.

The programming challenges it poses are daunting and stem directly from its
non-volatility, high-performance, and direct connection to the processor's
memory bus.  PM's raw performance demands the removal of system
software from the common-case access path, its non-volatility requires that (if
it is to be used as storage) updates must be robust in the face of system failures,
and its memory-like interface forces application software to deal directly with issues
like fault tolerance and error recovery rather than relying on layers of system software.

In addition, programming with PM exacerbates the impact of existing types of
bugs and introduces novel classes of programming errors.  Common
errors like memory leaks, dangling pointers, concurrency bugs, and data structure corruption have
permanent effects (rather than dissipating on restart).
New errors are also possible: A programmer might forget to log an update to a
persistent structure or create a pointer from a persistent data structure to
volatile memory.  The former error may manifest during recovery
while the latter is inherently unsafe since, after restart, the pointer to
volatile memory is meaningless and dereferencing it will result in (at best) an
exception.

The challenges of programming \emph{correctly} with PM are among the largest
potential obstacles to wide-spread adoption of PM and our ability to fully
exploit its capabilities.  If programmers cannot reliably write and modify code
that correctly and safely modifies persistent data structures, PM will be
hobbled as a storage technology.

Some of the bugs that PM programs suffer from have been the subject of
years of research and practical tool building.  The solutions and approaches to
these problems range from programming disciplines to improved library support
to debugging tools to programming language facilities.

% \remark{Some more elegant words}

Given the enhanced importance of memory and concurrency errors in PM
programming, it makes sense to adopt the most effective and reliable mechanisms
available for avoiding them.

The Rust programming language provides programming language-based mechanisms to
avoid a host of common memory and concurrency errors.  Its type system,
standard library, and ``borrow checker'' allow the Rust compiler to statically
prevent data races, synchronization errors, and most memory
allocation errors.  Further, the performance of the resulting machine code is
comperable with that of compiled C or C++.  In addition to these built-in
static checks, Rust also provides facilities that make it easy (and idiomatic)
to create new types of smart pointers that integrate cleanly with the rest of
the language.  Its type system also make data modification explicit and easy to control.

We have leveraged Rust's abilities to create \this{}, a library (or ``crate''
in Rust parlance) that lets programmers build persistent data structures.
\This{} provides basic PM programming facilities (e.g., opening and mapping
persistent memory pools) and a persistent software transactional memory
interface to provide atomic updates to persistent data.

Uniquely, \this{} uses static checking (in almost all cases) to enforce key PM
programming invariants:

\begin{itemize}
 
\item \This{} prevents the creation of unsafe pointers between non-volatile memory regions (or ``pools'') and pointers from those pools into volatile memory.
  
\item \This{} ensures that programs only modify persistent memory within transactions and that they log all updates to persistent state.
  
\item \This{} prevents most persistent memory allocation errors (e.g. dangling pointers, and multiple frees) in the presence of multiple, independent pools of PM.
  
%\item \This{} hides low-level PM programming operations (e.g. preserving write ordering, creating logs, etc.) behind its high-level interface.

\end{itemize}

Our experience building \this{} demonstrates the benefits that a strong type
system can bring to PM programming.  We evaluate \this{} quantitatively and
qualitatively by comparing it to existing PM programming libraries.  We find
that \this{} provides stronger guarantees than other libraries and that it is
as fast or faster than most of them.  We also highlight some changes
to Rust that would make \this{} even more powerful and efficient.

The rest of paper is organized as follows. \refsec{sec:background} gives a brief
background information on PM programming and the Rust language.
\refsec{sec:overview} describes \this{}.  In \refsec{sec:results}
we evaluate our library. \refsec{sec:related} places \this{} in context relative to related work.
 Finally, \refsec{sec:conclude} concludes.

% \remark{Mention lower level PM programming constructs}

% \remark{Mention learning curve shifting}



